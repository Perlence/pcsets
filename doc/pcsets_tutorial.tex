% $Id: pcsets_tutorial.tex 209 2007-08-19 16:41:44Z mccosar $
% --------------------------------------------------------------------

\documentclass[letterpaper,12pt,oneside]{book}

\usepackage{indentfirst}
\usepackage{verbatim}

\usepackage[T1]{fontenc}
\usepackage[english]{babel}

% LC 206
\usepackage[
  paper=letterpaper,
  tmargin=1in,
  bmargin=1in,
  includehead
]{geometry}

% LC 168
\usepackage{listings}
\lstset{
  language=Python,
  basicstyle=\tt
}
% LC 647
\usepackage{makeidx}
\makeindex
% LC 698
\usepackage{natbib}
\bibliographystyle{alpha}

\title{
  {\bf pcsets}:\\
  Pitch Class Sets for Python
}
\author{Bruce H. McCosar}

% --------------------------------------------------------------------

\begin{document}

\maketitle

\tableofcontents

% --------------------------------------------------------------------

\chapter{Introduction}

This tutorial is an introduction to two subjects at once:

\begin{itemize}
\item {\bf Pitch Class Sets}, and
\item the Python module {\bf pcsets}.
\end{itemize}

I'm hoping that the two will reinforce each other. That is, the
standard method of learning about Pitch Class Sets might be to read a
book\footnote{ Actually, there are some very good books. I've listed
the best in the {\bf References} section (p.~\pageref{references}).
I'll point out some of the most relevant as I go along. } or take a
class. Very few times do you have the opportunity to actually {\em
play} with them---to get some hands-on experience. Even if some
brilliant inspiration strikes, sometimes the aggravation of working
these set operations out manually can drag the idea to ground before
it even has the chance to fly high.

Well, here's an opportunity. A lot of the examples can be run in
Python interactive mode. You can not only learn the concept, you
can {\em try} the operation, immediately. You can develop your own
ideas or theories and prove or disprove them quickly. For the more
programming-oriented, I'm also including a few coding challenges that
make use of the material you're learning.

But above all, try some of these ideas out: {\em play} in the musical
sense. If you play an instrument, let these concepts be a springboard
to developing original music of your own. I'll give you some examples
of how I'd approach this, but keep in mind that I come from a jazz
background. You come from your own musical background, and you should
find your own path.


\section{Summary}

Pitch Class Sets are a mathematical model for analyzing and composing
music.\footnote{ The classic work of Pitch Class Set theory is Allen
Forte's {\em The Structure of Atonal Music} \cite{forte}. Straus has
also written a very readable introduction and overview \cite{straus}.
} Each note `C' through `B' has an equivalent pitch class number 0
through 11. Sets of these numbers may be operated on by mathematical
functions, leading to new combinations and creations.

The basic {\bf pcsets} modules free you from the drudgery of computing
Pitch Class Set operations by hand. Moreover, incorporated into a
programming language such as Python, the package allows application
of the Pitch Class Set concepts to the broader use of musical
interpretation and creation.


\section{Why Pitch Class Sets?}

Basically, one day, I got sick of II-V progressions. I'd read a lot on
chord-scale theory, but I wasn't satisfied that was the whole of the
musical universe. Sometimes I would play something, realize it worked,
then think, {\em Hmm---now what was that?} Standard music theory just
didn't cover it.\footnote{ This is something a lot of jazz musicians
run into, eventually. Mark Levine was the first author to bring it to
my attention \citep[p.~250]{levinep} with an unusual Herbie Hancock
chord, notated ``E$\flat{}7^{\flat{}9}$/F''. There aren't really any
standard scales having E$\flat$, E$\natural$, and F. }

Looking back through my music notebooks a few weeks ago, I found the
seed that started this all. On one page, I'd tried to work out a new
chord progression, then wrote a final thought on the page:

\begin{center}
{\em Is there a periodic table for chords?}
\end{center}

It turns out that there was. You'll read about it in a later section
(\S{} \ref{primecatalog}, p.~\pageref{primecatalog}). Since then, I've
found them to be a general use creative tool, not just a static table.


\section{Why pcsets?}

I decided to write this Python module after finding most of the
programs available online were GUI-only / interactive only (or worse
\ldots{} applets). For various reasons, I needed to be able to set up
long computational chains on a group of pitch class sets. Typing them
into a web browser one at a time and poking buttons was {\em not} an
option!

I released the module to the public under the GPL (Appendix \ref{gpl},
p.~\pageref{gpl}) for three reasons:

\begin{enumerate}
\item It might serve as an educational tool for music theory and
Pitch Class sets.
\item The addition of functions to connect set theory to the more
traditional chord / scale theory might lead to innovative, new
types of music software.
\item There wasn't a module available that provided the same
functionality.
\end{enumerate}


\section{The ``Uh Oh'' Section}

I'm assuming, if you're reading this, that you're familiar with:

\begin{itemize}
\item At least a little music theory;
\item the programming language Python; and
\item mathematical concepts such as {\em function} and {\em identity}.
\end{itemize}

I'm also assuming you've read the distribution's README and INSTALL
files, and have the {\bf pcsets} package installed on your system
properly. I mean, after all---considering the range of the above
requirements, you must be pretty good at whatever you do!\footnote{
All six of you who are reading this, that is. $\ddot\smile$ }


% --------------------------------------------------------------------

\chapter{The Basics}

\section{Pitch Classes}

Pitch Class values are the ``atoms'' of the {\bf pcset} world.
However, they do not cover every possible musical situation.


\subsection{Restrictions}
\index{pitch class!values!restrictions}

In the literature, and in the {\bf pcsets} package, Pitch Class values
are defined for {\em 12 tone equal temperament} only. \index{equal
temperament}

There are, of course, other methods of intonation, and other
intonation systems around the world. In fact, a PcSets module
tweaked for some sort of microtonal\index{microtonal} music might be
pretty interesting.\footnote{ David Lewin demonstrated this concept
quite well in Chapter 2 of his book \cite{lewin}, which is highly
recommended reading if you are more into the mathematics of set
theory. }

But not today.


\subsection{Definition}\index{pitch class!values!definition}

A {\em Pitch Class} is a single integer that represents a particular
musical note on the 12-tone equal temperament scale. `C' is defined as
zero\index{pitch class!values!zero}; ascending the scale, each note is
assigned consecutive integers until `B' (11).


\subsection{Enharmonic Equivalence}
\index{enharmonic equivalence}
\index{equivalence!enharmonic}

Because of the equal temperament basis, it is reasonable
that enharmonic equivalance translates to {\em exact} Pitch
Class equivalence. \index{pitch class!values!equivalence}
\index{equivalence!pitch class} No distinction is made between
C$\sharp$ and D$\flat$: they both have Pitch Class value~1.


\subsection{Octave Equivalence}
\index{octave!equivalence}
\index{equivalence!octave}

Perhaps the only Pitch Class value rule that may seem strange to a
musician is the principle of {\bf octave equivalence}. Any note with
the name `C'---whether it be played on the fifth string of a 7/8 scale
upright bass or the highest key on a piano---is considered to be pitch
class zero.

Musicians are accustomed to thinking of notes being in a particular
order, or in particular positions relative to other notes. Using Pitch
Class values, the ``relativity'' information is lost.

Pitch Classes, therefore, are an {\em abstraction} of musical
structures. \index{pitch class!values!as abstraction} The information
which we lose here is not really ``gone''. A musical structure
translated to pitch class values has been reduced to some sort of
essential structure; if we make any alterations to the structure, {\em
we}, human beings, have to {\bf construct a new meaning} from the
result.

I wish to emphasize this as a particularly important point. We will
see a lot of pitch class set transformations later on. Taken at face
value, {\em they mean nothing}. They only acquire meaning through our
interpretation.\index{pitch class!sets!meaning}

I can write a Python module to do the math, but it's up to the user to
provide the magic.


\subsection{Pitch Class Value Table}
\index{pitch class!values!table}

The above rules lead to a fairly simple note name to Pitch Class
conversion table:

\begin{center}
\begin{tabular}{|c|c|c|c|c|c|c|c|c|c|c|c|c|}
\hline
\rule[-3mm]{0mm}{8mm}
{\bf note name} &
C &
$\frac{\mathrm{C}\sharp}{\mathrm{D}\flat}$ &
D &
$\frac{\mathrm{D}\sharp}{\mathrm{E}\flat}$ &
E &
F &
$\frac{\mathrm{F}\sharp}{\mathrm{G}\flat}$ &
G &
$\frac{\mathrm{G}\sharp}{\mathrm{A}\flat}$ &
A &
$\frac{\mathrm{A}\sharp}{\mathrm{B}\flat}$ &
B \\
\hline
\rule[-3mm]{0mm}{8mm}
{\bf pitch class} &
0 & 1 & 2 & 3 &
4 & 5 & 6 & 7 &
8 & 9 & 10 & 11 \\
\hline
\end{tabular}
\end{center}

Please note, however, one simplification, which will be consistent
throughout the {\bf pcsets} module: I have omitted from the table
any ``accidental'' notes that have a natural equivalent. There is no
F$\flat$ on the table; E is always E. Similarly, there are no double
flats or double sharps.

I call this the {\em natural rule}. \index{pitch class!values!natural
rule} There is no particular mathematical basis for it; in fact, there
is no particular reason to have or use note names at all, other than
convenience. For this reason, the core \lstinline{pcsets.pcset} module
is a separate entity from the module which deals with traditional note
names (\lstinline{pcsets.noteops}).\index{pcsets!noteops}


\begin{comment} %%%%%%%%%%%%%%%%%%%%%%%%%%%%%%%%%%%%%%%%%%%%%%%%%%%%%%

\section{Pitch Class Sets}

Text.

\subsection{Definition}

Text.

\subsubsection{Components}

Text.

\subsubsection{Unique Membership}

Text.

\subsubsection{Unordered Set}

Text.

\subsection{Set Names}

Text.

\subsubsection{Special Sets}

Null
Unison
Interval
Chromatic

\subsubsection{General Sets}

Trichords
Tetrachords
Pentachords
Hexachords
Septachords
Octachords
Nonachords


\section{Representation}

Text.

\subsection{As a List}

Text.

\subsection{As a Specification String}

Text.

\subsection{As a String of Notes}

Text.

\subsubsection{noteops.pcfor}

Text.

\subsubsection{noteops.notes}

Text.

\subsubsection{Minimum Conflict}

Text.

\section{Transposition}

Text.

\subsection{Mod 12 Mathematics}

Text.

\subsection{Tn}

Text.

\subsection{T-n}

Text.

\subsection{TnTn}

Text.

\section{Inversion}

Text.

\subsection{I()}

Text.

\subsection{Other Inversions}

Text.

\subsection{Ixy(x,y)}

Text.

\section{Combined Operations}

Text.

\subsection{Inversion, then Transposition}

Text.

\subsection{Transposition, then Inversion}

Text.

\subsection{Multiple Operations}

Text.

\subsection{The ``Zorro Box''}

Text.

% --------------------------------------------------------------------

\chapter{The Standards}

\section{Normal Form}

Text.

\section{Zero}

Text.

\section{Reduced Form}

Text.

\section{Prime Form}

Text.

\subsection{Generating a Prime Family}

Text.

\subsection{Set Equivalence}

Text.

Tn or TnI

\subsection{Exact Equivalence}

Text.

Rearrangement

\section{The Prime Catalog}\label{primecatalog}

Text.

\subsection{Controversy}

Text.

\subsubsection{Major vs. Minor}

Text.

\subsubsection{Relativity for PcSets}

Text.


% --------------------------------------------------------------------

\chapter{Single Set Operations}

Text.


\section{Permutations}

Text.

\subsection{Reverse}

Text.

\subsection{Sort}

Text.

\subsection{Shift}

Text.


\section{Derivations}

Text.

\subsection{Complement}

Text.

\subsection{Interval Vector}

Text.

\subsection{Common Tone Vector}

Text.

% --------------------------------------------------------------------

\chapter{Operations on Two Sets}

pcops.

% --------------------------------------------------------------------

\chapter{Tone Rows}

tonerows.

% demo section -- cut here v v v

\chapter{Listing}\index{listing}

\begin{lstlisting}
>>> for x in range(20):
>>>     print x
\end{lstlisting}

More examples.

\chapter{Graphics}\index{graphics}

Example.

More examples.

% demo section -- cut here ^ ^ ^

\chapter*{About the Author}

All sorts of fun facts.


\appendix

\chapter{The GNU Public License}\label{gpl}

[Refer to license in main.]

\chapter{The GNU Free Documentation License}\label{gfdl}

[Too big, and formatted with all sorts of weird, nonstandard crap.]

\end{comment} %%%%%%%%%%%%%%%%%%%%%%%%%%%%%%%%%%%%%%%%%%%%%%%%%%%%%%%%


\clearpage
\addcontentsline{toc}{chapter}{References}\label{references}
\bibliography{pcsets}

\clearpage
\addcontentsline{toc}{chapter}{Index}
\printindex


\end{document}
